% Options for packages loaded elsewhere
\PassOptionsToPackage{unicode}{hyperref}
\PassOptionsToPackage{hyphens}{url}
%
\documentclass[
]{article}
\usepackage{amsmath,amssymb}
\usepackage{lmodern}
\usepackage{iftex}
\ifPDFTeX
  \usepackage[T1]{fontenc}
  \usepackage[utf8]{inputenc}
  \usepackage{textcomp} % provide euro and other symbols
\else % if luatex or xetex
  \usepackage{unicode-math}
  \defaultfontfeatures{Scale=MatchLowercase}
  \defaultfontfeatures[\rmfamily]{Ligatures=TeX,Scale=1}
\fi
% Use upquote if available, for straight quotes in verbatim environments
\IfFileExists{upquote.sty}{\usepackage{upquote}}{}
\IfFileExists{microtype.sty}{% use microtype if available
  \usepackage[]{microtype}
  \UseMicrotypeSet[protrusion]{basicmath} % disable protrusion for tt fonts
}{}
\makeatletter
\@ifundefined{KOMAClassName}{% if non-KOMA class
  \IfFileExists{parskip.sty}{%
    \usepackage{parskip}
  }{% else
    \setlength{\parindent}{0pt}
    \setlength{\parskip}{6pt plus 2pt minus 1pt}}
}{% if KOMA class
  \KOMAoptions{parskip=half}}
\makeatother
\usepackage{xcolor}
\usepackage[margin=1in]{geometry}
\usepackage{graphicx}
\makeatletter
\def\maxwidth{\ifdim\Gin@nat@width>\linewidth\linewidth\else\Gin@nat@width\fi}
\def\maxheight{\ifdim\Gin@nat@height>\textheight\textheight\else\Gin@nat@height\fi}
\makeatother
% Scale images if necessary, so that they will not overflow the page
% margins by default, and it is still possible to overwrite the defaults
% using explicit options in \includegraphics[width, height, ...]{}
\setkeys{Gin}{width=\maxwidth,height=\maxheight,keepaspectratio}
% Set default figure placement to htbp
\makeatletter
\def\fps@figure{htbp}
\makeatother
\setlength{\emergencystretch}{3em} % prevent overfull lines
\providecommand{\tightlist}{%
  \setlength{\itemsep}{0pt}\setlength{\parskip}{0pt}}
\setcounter{secnumdepth}{-\maxdimen} % remove section numbering
\ifLuaTeX
  \usepackage{selnolig}  % disable illegal ligatures
\fi
\IfFileExists{bookmark.sty}{\usepackage{bookmark}}{\usepackage{hyperref}}
\IfFileExists{xurl.sty}{\usepackage{xurl}}{} % add URL line breaks if available
\urlstyle{same} % disable monospaced font for URLs
\hypersetup{
  pdftitle={Measuring Term Premia for the Euro Area},
  hidelinks,
  pdfcreator={LaTeX via pandoc}}

\title{Measuring Term Premia for the Euro Area}
\author{}
\date{\vspace{-2.5em}Dev Prakash Srivastava}

\begin{document}
\maketitle

\textbf{INTRODUCTION}

This project aims to compute estimates of term premia for the Euro Area
- specifically for the countries Germany, France, Spain and Italy. We
will first give empirical estimates for the term premia for the
historical series beginning from 2000 to the present. For that we use
two approaches - predictive model and consensus forecast (will be
explained later). The aim is to estimate the term premia for the present
long term Euro area bonds, and also to allow the user to enter the
suitable parameters/forecasts to obtain the term premia.

The following sections detail the methodology used for the computation
of the estimates.

\textbf{THEORETICAL FRAMEWORK}

We first underline the theoretical framework used.

Consider zero-coupon bonds and apply the no arbitrage condition to a
one-period bond (the safe asset) and a T-period bond:

\begin{eqnarray*}
    E_{t}\left( r_{t,t+1}^{T} - r_{t,t+1}^{1}\right) = E_{t}\left(
        r_{t,t+1}^{T}-y_{t,t+1}\right) = \phi _{t,t+1}^{T}
\end{eqnarray*}

\begin{eqnarray*}
        E_{t}\left( r_{t,t+1}^{T}\right) = y_{t,t+1}+\phi _{t,t+1}^{T}
\end{eqnarray*}

Solving forward the difference equation
\(p_{t,T}=p_{t+1,T}-r_{t,t+1}^{T},\) we have :

\begin{eqnarray*}
        y_{t,T} = \frac{1}{\left( T-t\right) }\underset{i=0}{\overset{T-1}{\sum }}
        E_{t}\left( r_{t+i,t+i+1}^{T}\right)
        =\frac{1}{\left( T-t\right) }\underset{i=0}{\overset{T-1}{\sum }}
        E_{t}\left( y_{t+i,t+i+1}+\phi _{t+i,t+i+1}^{T}\right)
\end{eqnarray*}

So we have the following equation for the term spread

\begin{eqnarray*}
    y_{t,T}-y_{t,t+1}= \underset{i=1}{\overset{T-t-1}{\sum}}\left(1-\frac{i}{T-t}\right)E_{t}\Delta y_{t+i,t+i+1}+\frac{1}{T-t}\underset{i=1}{\overset{T-t-1}{\sum }}\phi _{t+i,t+i+1}^{T}
\end{eqnarray*}

\textbf{PROCEDURE}

We can use the term spread equation given above to estimate the term
premia. We use the common short-rate for the Euro Area, and the yields
on the 10-year government bond yields for the term spread. But the
future expected future path of short-term rates is not known and thus we
need some approach to estimate \(\Delta y_{t+i,t+i+1}\). Once we have
it, the term premia can be easily calculated.

To estimate the future path of short-term rates we use two approaches -

\begin{enumerate}
\def\labelenumi{\arabic{enumi}.}
\item
  Predictive Model - \begin{eqnarray*}
  \Delta y_{t+1,t+2} = \rho\Delta y_{t,t+1} + \epsilon_{t+1}
  \end{eqnarray*} We can estimate \(\rho\) using the historical data
  available. This parameter (assuming it be constant over time) can be
  used to get the future path of interest rates.While basic it can serve
  as an excellent baseline model.
\item
  Consensus forecasts -\\
  The ECB's Professional Forecasters Survey gives the average forecasts
  for the expected future short-term rates. These can be directly used
  and compared with our baseline model estimates.
\end{enumerate}

\end{document}
